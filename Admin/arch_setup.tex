\documentclass[a4paper, 12pt]{article}
\usepackage{textmpl_base}

\title{How-to install Arch Linux}
\author{Alexey Minchakov}
\date{\today}

\begin{document}
\maketitle

\section{Prepare flash drive}


\section{Prepare BIOS}
Security \textrightarrow{} Secure Boot \textrightarrow{} Disable
Also were disabled WWAN, Thunderbolt 3, Card Reader.

\section{Installation}

Setup internet connection:
\begin{lstlisting}
$> ping google.com
$> wifi-menu
\end{lstlisting}

Update \path{/etc/pacman.d/mirrorlist}, retrieve most up-to-date mirrors:
\begin{lstlisting}
$> pacman -Syy
$> pacman -S reflector
$> reflector -c "Russia" -f 12 -l 10 -n 12 --save /etc/pacman.d/mirrorlist
\end{lstlisting}

Разбивка диска с помощью \url{cfdisk} или \url{gdisk}.
Нужно установить раздел \url{UEFI} - выделить минимум 550 Мб в начале диска, тип диска должен быть \url{EFI boot partition}, отформатировать его  в \url{Fat32}.
Создаем новые разделы:
\begin{itemize}
\itemsep0em
\item \path{/boot} 500MB, ext2
\item \path{swap} размер RAM
\item \path{/} не менее 30GB ext4
\item \path{/var} 8GB ext4 - не обязательный раздел
\item \path{/home} все остальное
\end{itemize}

Check and prepare disk drive.
\begin{lstlisting}
$> fdisk -l # list disk partitions
$> mkfs.ext2 /dev/partition -L boot
$> mkfs.ext4 /dev/partition -L root
$> mkswap /dev/partition -L swap
$> pacman -S dosfstools vim
$> mkfs.fat -F32 /dev/partition
\end{lstlisting}

Mounting drives:
\begin{lstlisting}
$> lsblk # list block devices
$> mkdir /mnt/{boot,var,home}
$> mkdir /mnt/boot/efi
$> mount /dev/sdb1 /mnt/...
$> swapon /dev/sdb5
\end{lstlisting}

Begin installation:
\begin{lstlisting}
$> pacstrap -i /mnt base base-devel # install group of packages
$> genfstab -U -p /mnt >> /mnt/etc/fstab # generate /etc/fstab file
$> arch-chroot /mnt /bin/bash # change root dir
$> vim /etc/locale.gen    # uncomment lines en_US.UTF-8, ru_RU.UTF-8
$> locale-gen # generate locale files
# select timezone
$> ln -sf /usr/share/zoneinfo/Europe/Moscow /etc/localtime
# set hardware clock from system clock and update /etc/adjtime
$> hwclock --systohc [--utc]
$> echo MyHostName > /etc/hostname
$> vim /etc/hosts
# add lines:
# 127.0.0.1    localhost.localdomain    localhost
# ::1          localhost.localdomain    localhost
# 127.0.0.1    localhost.localdomain    MyHostName
$> systemctl enable dhcpcd
$> passwd     # setup root passwd
$> pacman -S grub [os-prober]
\end{lstlisting}

Now for legacy BIOS:
\begin{lstlisting}
$> grub-install --target=i386-pc --recheck /dev/partition
\end{lstlisting}

For UEFI:
\begin{lstlisting}
$> pacman -S efibootmgr
$> grub-install --targer=x86_64-efi --efi-directory=/boot/efi --bootloader-id=GRUB
\end{lstlisting}

Continue installation:
\begin{lstlisting}
> grub-mkconfig -o /boot/grub/grub.cfg
$> exit
$> umount -R /mnt
$> shutdown now
\end{lstlisting}

After reboot remove USB flash drive, login as root, add and configure new user:
\begin{lstlisting}
$> useradd -m -g users -G wheel -s /bin/bash UserName
$> passwd aim
$> EDITOR=vim visudo
# uncomment line %wheel=ALL (ALL) ALL
$> exit
\end{lstlisting}

Login as UserName
% login as aim
% > sudo pacman -S pulseaudio pulseaudio-alsa xorg xorg-xinit i3-gaps rxvt-unicode i3blocks dialog wpa_supplicant
% dialogs wpa_supplicant = для wifi
% перезагрузка. Если не работает wifi-menu, нужно сбросить параметры:
% > sudo netctl status wl...
% > sudo systemctl stop dhcpcd.service
% > sudo systemctl disable dhcpcd.service
% > sudo rm -rf /var/lib/dhcpcd/*.lease
% > sudo rm /etc/systemd/system/multi-user.target.wants/netctl*
% > sudo rm /etc/netctl/wl...
% > reboot
% > wifi-menu
% > sudo netctl start wl..
% > sudo netctl enable wl...
% в Admin находится конфиг файл для настройки скрытой wi-fi сети
% > !! некоторые пакеты зависят от networkmanager пакета, после его установки сети отваливается!
%
% -- скопировать мой dotfiles
% > git init .
% > git remote add -t \* -f origin https://github.com/lexaaim/dotfiles.git
% > git checkout master
%
% Установка oh-my-zsh
% > mv .oh_my_zsh 1
% > sh -c "$(curl -fsSL https://raw.github.com/robbyrussell/oh-my-zsh/master/tools/install.sh)"
% > mv 1/themes/... .oh-my-zsh/themes/
% > git checkout -- .zshrc
%
% Установка vim-plug:
% curl -fLo ~/.vim/autoload/plug.vim --create-dirs  https://raw.githubusercontent.com/junegunn/vim-plug/master/plug.vim
% > vim
% :PlugInstall
%
% Для того чтобы автоматически запускался startx (он прописан в .zprofile) нужно поставить шелл по умолчанию zsh :
% > chsh
%
% ШРИФТЫ:
% AUR > git clone https://aur.archlinux.org/ttf-dejavu-sans-mono-powerline-git.git
%
% темы иконки:
% > pacman -S arc-gtk-theme arc-icon-theme lxappearance
%
% Настройка времени
%
% монтирование tmpfs:
% проверить df -h
% tmpf 1000M на ~/.cache
%
% настройка яркости экрана: xorg-xbacklight + pacman -S xf86-video-intel
%
% скриншот - pacman -S scrot
%
% for disabling TrackPoint
% > xinput # show input devices
% > xinput --disable "Device Name"

\end{document}
