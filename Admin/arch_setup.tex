\documentclass[a4paper, 12pt]{article}
\usepackage{include/base}

\title{How-to install \url{Arch Linux}}
\author{Alexey Minchakov}
\date{\today}

\begin{document}
\maketitle
\tableofcontents

%===========================================
\section{Preparing to install}
\subsection{USB}

Download \path{linux.iso}.
\begin{lstlisting}
$> lsblk
$> sudo gparted                             # format flash Fat32
$> sudo dd bs=16M if=linux.iso of=/dev/sdc status=progress oflag=sync
\end{lstlisting}

\subsection{BIOS}

\menu{Security > Secure Boot > Disable}
\\Also were disabled WWAN, Thunderbolt 3, Card Reader.

%===========================================
\section{Installation}
\subsection{Internet settings}

Setup internet connection:
\begin{lstlisting}
$> ping google.com
$> wifi-menu
\end{lstlisting}

Update \path{/etc/pacman.d/mirrorlist}, retrieve most up-to-date mirrors:
\begin{lstlisting}
$> pacman -Syy
$> pacman -S reflector
$> reflector -c "Russia" -f 12 -l 10 -n 12 --save /etc/pacman.d/mirrorlist
\end{lstlisting}

\subsection{Disk patitioning}

Disk partition utils: \url{cfdisk}, \url{gdisk}.
Add partition \url{UEFI} - minimum 550MB at the beginning of disk, file system type should be \url{EFI boot partition}. Format it \url{Fat32}.
\\Partition table for Linux:
\begin{itemize}
\item \path{/boot} 500MB, ext2
\item \path{swap} RAM size
\item \path{/} minimum 30GB ext4 (depend on existing \path{/var} partition)
\item \path{/var} 8GB ext4
\item \path{/home} ext4
\end{itemize}

Check and prepare disk drive.
\begin{lstlisting}
$> fdisk -l                                 # list disk partitions
$> cfdisk /dev/sdv                          # setup partitions
$> gdisk                                    # setup partitions
$> mkfs.ext2 /dev/partition -L boot
$> mkfs.ext4 /dev/partition -L root
$> mkswap /dev/partition -L swap
$> pacman -S dosfstools vim
$> mkfs.fat -F32 /dev/partition
\end{lstlisting}

Mounting drives:
\begin{lstlisting}
$> lsblk                                    # list mount points
$> mkdir /mnt/{boot,var,home}
$> mkdir /mnt/boot/efi
$> mount /dev/sdb1 /mnt/...
$> swapon /dev/sdb5
\end{lstlisting}

\subsection{Installation process}

Begin installation:
\begin{lstlisting}
$> pacstrap -i /mnt base base-devel         # install group of packages
$> genfstab -U -p /mnt >> /mnt/etc/fstab    # generate /etc/fstab file
$> arch-chroot /mnt /bin/bash               # change root dir
$> vim /etc/locale.gen
\end{lstlisting}

\begin{lstlisting}[style=style_file,caption={/etc/locale.gen uncomment:}]
en_US.UTF-8
ru_RU.UTF-8
\end{lstlisting}

\begin{lstlisting}
$> locale-gen                               # generate locale files
# select timezone
$> ln -sf /usr/share/zoneinfo/Europe/Moscow /etc/localtime
# set hardware clock from system clock and update /etc/adjtime
$> hwclock --systohc [--utc]
$> echo MyHostName > /etc/hostname
$> vim /etc/hosts
\end{lstlisting}

\begin{lstlisting}[style=style_file,caption={/etc/hosts add lines:}]
127.0.0.1    localhost.localdomain    localhost
::1          localhost.localdomain    localhost
127.0.0.1    localhost.localdomain    MyHostName
\end{lstlisting}

\begin{lstlisting}
$> systemctl enable dhcpcd
$> passwd                                   # setup root passwd
$> pacman -S grub [os-prober]
\end{lstlisting}

If legacy BIOS:
\begin{lstlisting}
$> grub-install --target=i386-pc --recheck /dev/partition
\end{lstlisting}

If UEFI:
\begin{lstlisting}
$> pacman -S efibootmgr
$> grub-install --targer=x86_64-efi --efi-directory=/boot/efi --bootloader-id=GRUB
\end{lstlisting}

Continue installation:
\begin{lstlisting}
$> grub-mkconfig -o /boot/grub/grub.cfg
$> exit
$> umount -R /mnt
$> shutdown now
\end{lstlisting}

After reboot remove USB flash drive, login as root, add and configure new user:
\begin{lstlisting}
$> useradd -m -g users -G wheel -s /bin/bash UserName
$> passwd UserName
$> EDITOR=vim visudo
\end{lstlisting}

\begin{lstlisting}[style=style_file,caption={sudoers uncomment:}]
%wheel=ALL (ALL) ALL
\end{lstlisting}

\begin{lstlisting}
$> exit
\end{lstlisting}

Login as UserName:
\begin{lstlisting}
$> sudo pacman -S pulseaudio pulseaudio-alsa xorg xorg-xinit i3-gaps rxvt-unicode i3blocks dialog wpa_supplicant lua feh rofi playerctl gvfs mpg123 ntfs-3g
\end{lstlisting}

\subsection{WiFi setup}
If WiFi doesn't work (\url{netctl} may conflict with \url{networkmanager}):
\begin{lstlisting}
$> ls /sys/class/net                        # all network interfaces
$> sudo netctl status wl..
# $> sudo systemctl stop dhcpcd.service
# $> sudo systemctl disable dhcpcd.service
$> sudo rm -rf /var/lib/dhcpcd/*.lease
$> sudo rm /etc/systemd/system/multi-user.target.wants/netctl*
$> sudo rm /etc/netctl/wl..
$> reboot
$> wifi-menu
# $> cp Admin/etc/netctl/wifi /etc/netctl/wl..
$> sudo netctl start wl..
$> sudo netctl enable wl..
\end{lstlisting}

\subsection{Autostart Xorg}
Change default shell (for \url{X} autostart from \path{$HOME/.zprofile} script):
\begin{lstlisting}
$> chsh
\end{lstlisting}

%===========================================
\section{Main settings(dotfiles, desktop)}

To get all stuff:
\begin{lstlisting}
$> git init .
$> git remote add -t \* -f origin https://github.com/lexaaim/dotfiles.git
$> git checkout master                      # get all branches and co master
\end{lstlisting}

\subsection{Oh My ZSH}
Setup \url{Oh My ZSH!}:
\begin{lstlisting}
$> mv .oh_my_zsh bkp                        # backup
$> sh -c "$(curl -fsSL https://raw.github.com/robbyrussell/oh-my-zsh/master/tools/install.sh)"
$> mv bkp/themes/... .oh-my-zsh/themes/
$> rm -rf bkp
$> git checkout -- .zshrc
\end{lstlisting}

\subsection{Vim}
Setup \url{Vim}:
\begin{lstlisting}
$> curl -fLo ~/.vim/autoload/plug.vim --create-dirs  https://raw.githubusercontent.com/junegunn/vim-plug/master/plug.vim
vim> :PlugInstall
\end{lstlisting}

\subsection{Fonts, icons}
\begin{lstlisting}
$> cd Temp/AUR
$> sudo pacman -S lxappearance

# font: Dejavu Sans Mono Powerline
$> git clone https://aur.archlinux.org/ttf-dejavu-sans-mono-powerline-git.git
$> cd ttf-dejavu-sans-mono-powerline-git
$> ./ttf-dejavu-sans-mono-powerline-git.install

# font: Yosemite San Francisco
$> git clone https://github.com/supermarin/YosemiteSanFranciscoFont
$> cp YosemiteSanFranciscoFont/*.ttf ~/.fonts
$> ~/.fonts/update-fonts.sh

# icons: Arc theme
$> pacman -S arc-gtk-theme arc-icon-theme
$> lxappearance
\end{lstlisting}

\subsection{Root's .vimrc}
Use user's \url{.vimrc} settings as Root:
\begin{lstlisting}
$> sudo ln -s /home/UserName/.vimrc /root/.vimrc
$> sudo ln -s /home/UserName/.vim/ /root/.vim
\end{lstlisting}

\subsection{Ethernet}
Manually setup ethernet:
\begin{lstlisting}
$> systemctl enable dhcpcd
$> ls /sys/class/net
$> ip link show dev emp0s3..
$> sudo ip link set enp0s3.. up
\end{lstlisting}

\subsection{Setup keyboard layouts switching}
Change keyboard layout with \keyss{\capslock}:
\begin{lstlisting}
$> cp ~/Admin/etc/X11/xorg.conf.d/00-keyboard.conf /etc/X11/xorg.conf.d/
$> reboot
$> cd Temp/AUR
$> git clone https://aur.archlinux.org/xkb-switch-git.git
$> cd xkb-switch-git
$> makepkg -si
\end{lstlisting}

\subsection{Setup LightDM and Greeter}
Setup \url{lightdm}
\begin{lstlisting}
$> sudo pacman -S lightdm lightdm-gtk-greeter
$> cp ~/Admin/etc/lightdm/lightdm.conf /etc/lightdm/
$> systemctl enable lightdm.service
# greeter autologin:
$> groupadd -r autologin
$> gpasswd -a UserName autologin
\end{lstlisting}

\subsection{List installed packages}
\begin{lstlisting}
$> sudo pacman -S scrot \                   # take screenshots
    vlc gstreamer \                         # codecs
    texlive-most texlive-lang evince biber\ # latex
    gvfs thunar mc ntfs-3g                  # automount, fs, filemanager
\end{lstlisting}

%===========================================
\section{Hardware and Power management}

\subsection{SSD setup}
Setup SSD Trim attributes in \path{/etc/fstab}:
\begin{itemize}
\item ''\url{discard}'' to \url{ext4}, \url{swap} fs
\item ''\url{noatime}'' to \url{ext2}, \url{ext4} fs
\end{itemize}

\subsection{Mount directories in tmpfs}
Mount tmpfs:
\begin{lstlisting}
$> df -h
# example /etc/fstab tmpfs 1000M ~/.cache:
$> sudo vim /etc/fstab
\end{lstlisting}

\begin{lstlisting}[style=style_file,caption={/etc/fstab add line:}]
tmpfs  /home/user/.cache  tmpfs  defaults,nodev,nosuid,mode=1777,size=1G  0 0
\end{lstlisting}

\subsection{Setup SWAPpiness}
\begin{lstlisting}
$> sudo cp ~/Admin/etc/sysctl.d/99-sysctl.conf /etc/sysctl.d/
$> reboot
\end{lstlisting}

\subsection{Keyboard setup Fn+F11}
To enable Lenovo Thinkpad \keyss{Fn, F11} and \keyss{Fn, F12} combinations:
\begin{lstlisting}
# show all keynames for i3
$> xmodmap -pke | grep -i backli
# To test you can use:
$> sudo evtest /dev/input/event5

$> sudo cp ~/Admin/etc/udev/hwdb.d/thinkpad_keyboard.hwdb /etc/udev/hwdb.d/
$> sudo udevadm hwdb --update
# it remaps 374 and 364 codes to 148 and 149
# for get which XF86 key it use check:
$> xbindkeys -k
\end{lstlisting}

\subsection{Disabling Thinkpad trackpoint or other input devices}
Disabling TrackPoint:
\begin{lstlisting}
$> xinput                                   # show input devices
$> xinput --disable "Device Name"
\end{lstlisting}

\subsection{Setup TLP power management}
\url{Tlp} setup (Arch wiki "Power management"):
\begin{lstlisting}
$> sudo pacman -S tlp                       # don't setup tlp-rdw
$> sudo tlp stat | less                     # read the recomendations
$> sudo cp ~/Admin/etc/default/tlp /etc/default/tlp/
$> systemctl enable tlp.service
$> systemctl enable tlp-sleep.service
$> systemctl mask systemd-rfkill.service
$> pacman -S tpacpi-bat                     # not tp_smapi
# after reboot check $> systemctl --failed
$> rkfill list                              # list all wireless devices
# profiles for devices
$> sudo cp ~/Admin/etc/modprobe.d/audio_powersave.conf /etc/modprobe.d/
$> sudo cp ~/Admin/etc/modprobe.d/webcamera_disable.conf /etc/modprobe.d/
$> sudo cp ~/Admin/etc/udev/rules.d/*.rules /etc/udev/rules.d/
\end{lstlisting}

\subsection{LCD Backlight}
\begin{lstlisting}
$> sudo pacman -S xorg-xbacklight xf86-video-intel
$> sudo cp ~/Admin/etc/X11/xorg.conf.d/10-backlight.conf /etc/X11/xorg.conf.d/
\end{lstlisting}

\subsection{Disable laptop interrupt}
If laptop undocking 100\% core:
\begin{lstlisting}
# for block interrupt (find it by test-disabling /sys/firmware/acpi/interrupts/)
$> echo
$> sudo cp ~/Admin/etc/systemd/system/disable_interrupt_gpe6F.service /etc/systemd/system/
$> systemctl enable disable_interrupt_gpe6F.service
\end{lstlisting}

%===========================================
\section{Additional software}

\subsection{.desktop files}
To add \path{*.desktop} file to application list:
\begin{lstlisting}
$> cp *.desktop ~/.local/share/applications/
\end{lstlisting}

\subsection{Setup FTP server}
\begin{lstlisting}
$> sudo pacman -S vsftpd
$> sudo cp ~/Admin/etc/vsftpd.conf /etc/
$> systemctl enable vsftpd
\end{lstlisting}

\subsection{VirtualBox installation}
\begin{lstlisting}
$> sudo pacman -S virtualbox virtualbox-host-modules-arch
# install Dynamic Kernel Module Support
$> sudo pacman -S dkms linux-headers
$> dkms status
$> dkms autoinstall                         # rebuild all kernel modules
$> sudo modprobe vboxdrv

# install USB 2.0 driver
$> cd ~/Temp/AUR
$> git clone https://aur.archlinux.org/virtualbox-ext-oracle.git
$> cd virtualbox-ext-oracle
$> makepkg- si

#install Guest Additions
$> sudo pacman -S virtualbox-guest-iso
$> cp /usr/lib/virtualbox/additions/VBoxGuestAdditions.iso ..
\end{lstlisting}

\subsection{Setup Docker}
Docker setup:
\begin{lstlisting}
# delete all docker containers
$> docker rm $(docker ps -a | grep -v "_data" | awk 'NR>1 {print $1}')
\end{lstlisting}

\end{document}
